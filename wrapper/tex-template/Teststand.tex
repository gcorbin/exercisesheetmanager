\documentclass[german, solutions, annotations]{fbmatheblatt}

\semester{WS51/52}
\lecture{Differentialgleichungen}
\sheetname{Blatt}
\sheetno{666}
\lecturer{Prof. Dr. L. Euler} 
\releasedate{01.01.1751}
\deadline{08.01.1751}

\begin{document}	
	\begin{sheet}	
			
		\begin{disclaimer}
			Dies ist ein Teststand um neue Aufgaben zu erstellen, bevor diese in den Pool eingefügt werden. 
		\end{disclaimer}
	
		\begin{homework}{Exponentielles Wachstum}{2 Punkte}
			Lösen sie die DGL $\dot x = x$ mit Anfangsbedingung $x(0) = 1$. 
		\end{homework}
		\begin{solution}
			Die allgemeine Lösung ist $x = c\exp(t)$. 
			Aus der Anfangsbedingung folgt $1 = c\exp(0) = c$, also ist die Lösung des Anfangswertproblems $x(t) = \exp(t)$. 
		\end{solution}
		\begin{annotation}
			1 Punkt für die allgemeine Lösung, 1 Punkt für die Bestimmung der Konstanten. 
		\end{annotation}
	\end{sheet}	
\end{document}