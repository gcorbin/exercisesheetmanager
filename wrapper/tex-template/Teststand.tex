\documentclass[Loesungen, Punkte]{fbmatheblatt}
%%%%%%%%%%%%%%%%%%%%%%%%%%%%%%%%%%%%%%%%%%%%%%%%%%%%%%%%%%%%%%%%%%%%%%%%%%%%%%%%
% latex-preamble for exercises: Einführung in das wissenschaftliche Programmieren (EWP), TU Kaiserslautern
% lecturer: Dr. Martin Bracke
% author: Matthias Andres
% email: andres@mathematik.uni-kl.de
% last update: October 20, 2018
%%%%%%%%%%%%%%%%%%%%%%%%%%%%%%%%%%%%%%%%%%%%%%%%%%%%%%%%%%%%%%%%%%%%%%%%%%%%%%%%

\usepackage[]{standalone}
\usepackage[utf8]{inputenc}
\usepackage[]{amsmath}
\usepackage[]{amsfonts}
\usepackage[]{amssymb}
\usepackage[]{amsthm}
\usepackage{bm}
\usepackage[]{eurosym}
\usepackage[]{graphicx}
\usepackage[]{subfig}
\usepackage[]{caption}
\usepackage[]{alltt}
\usepackage[]{colortbl}
\usepackage[]{stmaryrd}
\usepackage[]{enumerate}
\usepackage[]{tikz}
\usepackage[]{pgfplots}
\usepackage{nicefrac}
\usepackage{empheq}
\usepackage{makecell}
\usepackage{listings} % code embedding
\usepackage{enumerate} % change style of enumerate

% bibliography stuff
\usepackage{natbib}
\usepackage{bibentry}
\bibliographystyle{plain}
\nobibliography*

\usepackage{EWPStyle}

\newcommand{\pathToResource}{./}

\Semester{WS51/52}
\Veranstaltung{Differentialgleichungen}
\Blattname{Blatt}
\Blattnummer{666}
\Dozenten{Prof. Dr. L. Euler} 
\Erscheinungsdatum{01.01.1751}
\Abgabedatum{08.01.1751}

\begin{document}	
	\begin{Blatt}
		
		\begin{Disclaimer}
			Dies ist ein Teststand um neue Aufgaben zu erstellen, bevor diese in den Pool eingefügt werden. 
		\end{Disclaimer}
	
		\begin{Hausaufgabe}{Exponentielles Wachstum}{2 Punkte}
			Lösen sie die DGL $\dot x = x$ mit Anfangsbedingung $x(0) = 1$. 
		\end{Hausaufgabe}
		\begin{Loesung}
			Die allgemeine Lösung ist $x = c\exp(t)$. 
			Aus der Anfangsbedingung folgt $1 = c\exp(0) = c$, also ist die Lösung des Anfangswertproblems $x(t) = \exp(t)$. 
		\end{Loesung}
		\begin{Punkte}
			1 Punkt für die allgemeine Lösung, 1 Punkt für die Bestimmung der Konstanten. 
		\end{Punkte}
	\end{Blatt}

\nobibliography{literature}
	
\end{document}